% document de test for report-DMKM.cls (LaTeX2e)
% Created May 15, 2012

% We use the style given in rapport-DMKM.cls
\documentclass{report-DMKM}

% this package (optionnal) allows merging rows and columns in latex table 
\usepackage{amsmath}
\usepackage{multirow}
\usepackage{hyperref}
\usepackage{float}

%The following fields (title, author, date, tutors et place) must be present
\title{Mining parliamentary data and news articles \newline to find patterns of collaboration between politicians and third party actors}
\author{Francisco Andr\'es RODR\'IGUEZ DRUMOND}
% Defense date
\date{July 7th, 2014}
\tutors{\mbox{Marta Arias, LARCA-UPC},\newline
        Josep Larriba, DAMA-UPC\newline}

\place{Universitat Polit\`ecnica de Catalunya, Barcelona, Spain.}
% You can add the logo of the institution where you carry out your internship. Example here ERIC lab

%\logo{img/dama_logo.jpg}{25mm} 

\begin{document}

% Cover page (mandatory)
\maketitle

\newpage
% summary in English  (mandatory)
\begin{abstract}
There is an increasing interest in having tools to closely monitor the activity of politicians for transparency and accountability purposes. Among the different approaches in the literature, using Social Network Analysis tools for studying the dynamics of power inside a Parliament has yielded interesting results. However, finding collaborations between politicians and third party actors, which often do not appear in bills or in the transcripts of debates, remains a challenge. In this work we present a novel technique for automatically generating social networks relating politicians and other relevant third party actors that can be used for political analysis. Our proposal uses parliamentary data and news articles for automatically detecting relevant entities and discovering their relations by looking at their semantic similarity and co-occurrence in documents. We i) use bills approved by a parliament to model topics, ii) use these topics to find related news articles, iii) perform entity extraction and filtering and iv) compute and threshold entity-entity similarity measures to generate social networks describing the dynamics surrounding a bill. We apply our tool to analyze bills from the Catalan Parliament and show how relevant insights indicating patterns of collaboration between political actors can be gained.
\end{abstract}

% summary in French  (mandatory)
\begin{resume}
LOPEN IPSUM
\end{resume}

% table of content   (mandatory)
\tabledesmatieres

\section{Hosting institution}

DAMA-UPC, the DAta MAnagement group at Universitat Polit\`ecnica de Catalunya (UPC) is part of the Computer Architecture Department (DAC). The main research topics of DAMA-UPC are oriented to performance, exploration and quality in data management, focusing particularly on large data volumes. Specifically, they have investigated the creation of new data structures, algorithms, methods and applications in the area of Data Management that make it easier to manipulate large amounts of data.\\

DAMA-UPC is a member of Tecnio since 2005. Tecnio is an initiative of ACC10, the Agency for Innovation and Internationalization of the Catalan Enterprise, belonging to Generalitat de Catalunya. It is supported by Generalitat de Catalunya as a Consolidated Research Group (SGR-1187) and by the Ministry of Education and Science of Spain. Since its creation it has worked with important technological partners such as IBM, Oracle technologies, the Ministry of Cience and Innovation, the Health Department of the Generalitat de Catalunya, among others; and is current participating in four European Research Projects. 

Moreover, DAMA-UPC has been a pioneer in scientific collaboration, creating and organizing for three consecutive years the Workshop on Graph-based Technologies and Applications (Graph-TA), organizing the 17th International Database Engineering \& Applications Symposium (IDEAS 2013) and the First University-Industry Meeting on Graph Databases (UIM-GDB). The work of DAMA has also give birth to Sparksee, a graph database, and Sparsity Technologies.\\

\footnote{Text partially taken from \url{www.dama.upc.edu}}

\newpage
\section*{Acknowledgement}

To my parents, the reason I am here\\

To my supervisors Larri and Marta, for their faith in this project, for their inconditional support \\

Diego\\

To Matthew, Marcello, Caio and the rest of my friends who made DMKM the most pleasant of journies. \\

\newpage
\section{Introduction}\label{sec:intro}

Modern law-making is characterized by the indirect participation of third-parties in the legislative process to attempt to influence the contents and outcome of the discussions according to their interests. This process, known as \emph{lobbying}, is regularly done by companies, Non-Governmental Organizations (NGOs) and in some cases citizens and foreign governments. For instance, when a law concerning human rights is being discussed in a parliament we can expect human rights NGOs to contact politicians to attempt to push their agenda. \\

Most of the times citizens are completely unaware of the lobbying activities of their representatives. This is due to the fact that it is hard to closely monitor the activities of politicians, to understand the intricacies behind the drafting of bills and to grasp the implications they have on organizations and society. Knowing how decisions makers are influenced is however of great interest for politologists, journalists and voters in general. By tracking these relations we would be able to understand better the way decisions are made by government, have a better sense of who politicians are, who they serve and what their agenda is, and in some cases fight corruption. \\

Recently, there has been considerable work in the development of data mining tools to analyze the way legislative bodies work. Social Network Analysis (SNA) has been widely used to understand the dynamics of complex social systems and is a natural tool for modelling the dynamics of power in a parliament. There is a wealth of knowledge to be extracted from parliamentary data to model relationships between politicians, identify key players and sub-communities, predict the voting of bills and in general gain a better understanding of the legislative process. However, finding political relations between representatives and third party actors, who often do not appear in bills or in the transcripts of debates, remains a challenge. How can we automatically detect that a politician and a company might be aligned from information available to the public? \\

News articles contain a sizable amount of information about what happens in a given country and about its relevant actors. Because of this, there is plenty of work by the scientific community to develop methods to automatically extract useful knowledge from news articles. The development of applications to automatically generate Social Networks (SN) from corpora of news articles is particularly interesting in our context. This has been done extensively in the past in different areas including defense, counter-terrorism, news summarization and crime prevention. The working assumption is that by detecting patterns of co-occurrence of two entities in text we can establish that they are in some way related.  \\

In this study we use that assumption and formulate the hypothesis that by analyzing and combining parliamentary data and news articles we can generate graphs descriptive of the political closeness of politicians and relevant actors of a country. Our purpose is consequently to propose a method to automatically detect patterns that could be indicative of a lobbying activity. To do this, we automatically analyze news articles, text of bills discussed in a parliament and the transcripts of the debates. We show that it is possible to i) track the participation and the position of a politician with respect to a law and ii) find entities in news articles which could be directly related to the contents of a law. Laws thus can then be used as a cornerstone for detecting relationships between politicians and third-party actors. These relationships can then be used to build a SN which can be analysed using SNA tools. In this document we present our proposal and show the results obtained by its application in the context of the Parliament of Catalonia. \\

It is important to clarify that despite being motivated by the detection of patterns that can be suggestive of a lobbying process, in practice we aim to find relationships of political similarity or dissimilarity between two entities. The relations found by our method do not necessarily imply that the two actors are in direct liaison;  to do so we would need to closely monitor all the activities of politicians and organizations to verify with whom they are in contact. This is naturally unreasonable due to privacy considerations. \\

We aim to detect links between entities that indicate that they are both related to a particular political decision - in our case bills -, from which we could then establish political affinity or aversion. Naturally, if two entities have highly similar political views in a broad set of issues then one can believe that they could be collaborating. This could be verified by investigative journalists and by considering other sources of information like donations information, speeches, etcetera. Regardless of the verification of a lobbying process, having a graph relating politicians and third parties is in itself useful to understand the political landscape of a country. In this study we also show how our proposal is useful for political analysis, besides from the interest in the detection of lobbying activities. \\

The rest of this document is organized as follows: in chapter \ref{sec:related-works} we present the state of the art in the automatic generation of SN from corpora of text and in the use of SNA in the context of political science. In chapter\ref{sec:proposal} we present our proposal and in chapter \ref{sec:implementation} we present some implementation considerations deemed relevant. Next, we present the obtained results in section \ref{sec:results} along with some interesting applications of our method. Finally, we present in chapter \ref{sec:conclusions} the conclusions of our work along with some suggestions for further work.


\newpage
\section{Problem definition}\label{sec:problem_definition}

Social Networks (SN) are a powerful tool for understanding complex social systems, in which relationships between actors play a central role. Because of this, Social Network Analysis has become increasingly popular in the Political Science community for studying the dynamics of power. For instance, the authors of \cite{fowler2006connecting} studied a graph of co-sponsorship of bills to study interactions between congressmen in the US House of Representatives and found that by using network analysis tools it was possible to find highly influential politicians. Similarly, in \cite{kirkland2011relational} the authors  developed a theory of influence diffusion across a legislative network of relations based on weak and strong links and found patterns useful for determining the success of a bill.\\

The growing popularity of SNA in Politicial Science has given birth to Policy Network Analysis, a discipline concerned in the creation and analysis of SN involving political actors.\\

%In \cite{thomas2006get} we find a proposal to predict the voting of a bill based on speeches made by congressmen by detecting evidence of endorsements.  \\

\subsection{Policy Networks: Social Network Analysis for Political Science}\label{subsec:defining_pn}

\emph{Policy Network Analysis} (PNA) is the discipline in Political Science which focuses on the discovery and analysis of links between government and other members of a society, with a particular interest in the understanding of the policy making process and its outcomes. There are plenty of definitions of Policy Networks (PN) in the literature, basically because there are many ways to characterize relationships between actors of a society and several ways to approach the analysis. However the Oxford Handbook of Public Policy proposed a definition which is widely accepted and encompasses many of the other proposals. They state that a PN is a ``set of formal institutional and informal linkages between governmental and other actors structured around shared if endlessly negotiated beliefs and interests in public policy making and implementation''\cite{moran2008oxford}. \\

One of the main difficulties in PNA is that the generation of these PN is usually done in manual, cumbersome processes by experts. This involves the use of interviews, questionnaires and other instruments from the social sciences. During the actual generation of the network many subjective factors may come into play as the overall result depends on the people involved in the study. Consequently, PN generation involves a significant investment that does not always ``lead to breathe taking empirical and theoretical results'' \cite{kenis1991policy}. The question is consequently, how can we use technology to improve the process of PN identification. 

\subsection{Towards the automatic generation of PNs: defining our objectives}\label{subsec:objectives}

The main objective of this study is to propose a method for the automatic generation of PN surrounding the Legislative Branch of government. We decided to focus on the Legistative Power for two reasons. First, as we have previously established, Parliaments write laws and regulations, and thus have an enormous power in the shaping of a society. This makes them the primary target of lobbyists. \\ 

Second, as we will later see in section \ref{sec:related-works}, the challenges of automatic Social Network generation lie in i) understanding and characterizing the links found among the actors occurring in the SN and ii) discovering hidden relationships, which can only be detected by learning that two entities share a common topic.  When outlining this study, we aimed to define the problem in a way that we could address these two difficulties and produce SN which are meaningful, easy to interpret and with the highest number of possible relevant connections. \\

As we will see later on, bills can then be used as a cornerstone for detecting relationships between politicians and third-party actors. \\% We show that given a law it is possible to i) track the participation and the position of a politician with respect to it, ii) find entities in news articles which could be directly related to its content, iii) compute similarity measures between entities - and generate a graph - by taking into account their relationship to the bill iv) use the bill to label the discovered relationships.\\

The reader should know that for the rest of this document, we use the words bills and laws exchangeably to refer to the rules and regulations approved by a legislative body. We also use the terms entity, actors, political actors to refer to companies, non-governmental organizations, governments, advocacy groups, citizens and in general any person or association that is related to a bill, meaning that the contents of the bill affect their interests or policies. \\

\subsection{What type of PN do we aim to generate?}\label{subsec:objectives}

Among the different types of PN that we could study, we are concerned with the generation of two types of SN, or graphs. First, we are interested in generating SNs depicting the relationship of relevant actors with the bills approved by a legislative body. This SN, which we will call the \emph{Bill-Entity graph} is a di-graph with two types of nodes: bills and entities. In the \emph{Bill-Entity graph} graph there is a link between a bill and an entity if and only if the entity is related to the bill. This type of graph is useful for understanding how different actors relate to the bills, and for also understanding how the latter are interrelated.\\

Second, we are interested in generating, from the \emph{Bill-Entity graph}, an \emph{Entity-Entity graph} capturing possible relationships between two actors. These definition of this type of relationship is vague: we are interested in discovering pairs of actors that are i) related to a bill and ii) are politically affine or antagonistic, meaning that they could possibly know each other, be in contact and cooperate or compete to push their agendas. \\  

In many cases entities have either a positive or negative position with respect to a law. In the case of politicians, their voting history for a particular bill is recorded and usually accessible through Parliament websites or open data. In the case of companies, NGOs and other types of actors, their position could be assessed automatically, by employing sentiment analysis or other NLP techniques, or manually by experts. \\ 

This polarity can be used to annotate the links of the \emph{Bill-Entity graph} and the relationships of the \emph{Entity-Entity graph} by the following premise: two entities are positively related if they are positively or negatively related to a common bill; similarly, two entities are negatively related if they have different polarities with respect to a common bill. Naturally it might be the case that an actor does not have a well-defined relationship with respect to a law or that it is impossible to objectively classify it. Nonetheless, this schema allows for the generation of enriched graphs which allow for more refined graph analysis techniques. For instance, in \cite{traag2009community} we find a community detection algorithm which works with positive and negative links. 

\subsection{Stop Online Piracy Act (SOPA): a motivating example}\label{subsec:objectives}

\begin{figure}[h!]
    \centering
    \includegraphics[width=1\textwidth]{figs/sopa_bill_entity.png}
    \caption{Bill-entity graph for SOPA.}
    \label{fig:example_bill_entity}
\end{figure}

\begin{figure}[h!]
    \centering
    \includegraphics[width=1\textwidth]{figs/sopa_entity_entity.png}
    \caption{Entity-entity graph for SOPA.}
    \label{fig:example_entity_entity}
\end{figure}

As an illustrative  example, consider the \emph{Stop Online Piracy Act} (SOPA). SOPA was a bill introduced in 2011 the United States to combat online copyright infringement and online trafficking in counterfeit goods. If a website was found to infringe the law, SOPA allowed court orders to require Internet service providers to block access, prevent search engines from listing them and forbid advertising networks or other payment facilities from conducting business with the website. SOPA also made the unauthorized streaming of copyrighted content a criminal offense, imposing a penalty of up to five years in prison. \\ 

Due to it's controversy, SOPA was widely discussed internationally and many organizations raised their voices in favor and against the bill\footnote{A list of organization in favour and against can be found in \url{http://en.wikipedia.org/wiki/List_of_organizations_with_official_stances_on_the_SOPA_and_PIPA}}. In broad terms, organizations which rely on copyright strongly supported the bill. This includes, for instance:

\begin{itemize}
\item {\bf Pharmaceutical companies and associations}  like Pharmaceutical Research and Manufacturers of America (PhRMA), Pfizer, Alliance for Safe Online Pharmacies (ASOP). 
\item {\bf TV channels}: ABC, NBC, ESPN, Disney, among others. 
\item {\bf The music industry}, including the Recording Industry Association of America, the American Federation of Musicians, Sony Music Entertainment, Universal Music, etc.
\end{itemize}

Similarly, organizations that advocate for freedom and liberties, or whose business would be negatively affected by increased regulations voiced the opposition against the bill. Among these, we highlight:

\begin{itemize}
\item {\bf Torrent and streaming companies}: such as Pirate Bay, TorrentFreak, Vimeo, Grooveshark, Flickr, etc
\item {\bf Technological corporations}, such as Facebook, Yahoo!, Microsoft, Google.
\end{itemize}

Finally, we can also know the position of congressmen by looking at their voting history or their positions. In the case of SOPA, Republicans Lamar Smith and Mary Bono Mack, and Democrat Howard Berman were among the sponsors of the bill, while Democrat Ron Wyden and Republicans Marco Rubio, Rand Paul were against.\\

Based on this information, which we have manually gathered, we could generate a \emph{Bill-Entity graph} and an \emph{Entity-Entity graph} as the ones shown in figures \ref{fig:example_bill_entity} and \ref{fig:example_entity_entity}. These graphs were generated in a manual way taking into account the relationship of the entity with respect to the bill as defined before. \\

Note that the \emph{Bill-Entity graph} is not particularly useful by itself as it is only a way to visualize the information about a specific law. However if analyzed along with other bills, we can see how they overlap and understand for instance the similarity between two bills based on the entities that are related to them, the distance between two actors based on the whole set of legislations, compute centrality measures to detect influential politicians, etcetera. \\ 

The \emph{Entity-Entity graph} is useful on another hand for detecting communities of actors that share similar interests for a specific bill. Note that we deliberately chose not to draw edges between the politicians and the rest of the political actors; this is because we do not have the knowledge to suggest that a politician and a company might be related. We decided to draw only links between organizations of the same type, with the exception of the relationship between the Recording Industry Association of America and Grooveshark, TorrentFreak and PirateBay. Since the latter three are used to illegally download music we can infer that they are related. Our objective in this study is to produce a system that can automatically detect these and other obscure relationships automatically. \\

What type of analysis can we do with the \emph{Entity-Entity graph}? We can use the traditional SNA tools to individually analyze bills or alternatively look at the overlapping of these graphs accross different bills. This graph is however particularly useful for detecting possible lobbying relationships. For instance if a politician (or group of politicians) has a high number of links with the political actors of a specific community (imagine for instance that Lamar Smith had links with most of the Music Industry group) then we could possibly infer that there is a lobbying relationship.  \\


\subsection{Is it really possible to detect lobbying in an automated way? An important caveat}\label{subsec:really_lobbying} 

It is important to clarify that despite being motivated by the detection of patterns that can be suggestive of a lobbying process, in practice we aim to find relationships of political similarity or dissimilarity between two entities. The relations found by our method do not necessarily imply that the two actors are in direct liaison;  to do so we would need to closely monitor all the activities of politicians and organizations to verify with whom they are in contact. This is naturally unreasonable due to privacy considerations. \\

We aim to detect links between entities that indicate that they are both related to a particular political decision - in our case bills -, from which we could then establish political affinity or aversion. Naturally, if two entities have highly similar political views in a broad set of issues then one can believe that they could be collaborating. This could be verified by investigative journalists and by considering other sources of information like donations information, speeches, etcetera. Regardless of the verification of a lobbying process, our method is intended to be an unbiased, low-cost, semiautomated toal to aid the process of Policy Network generation and analysis. 


\newpage
\section{Related works}\label{sec:related-works}

There is an increasing interest in the development of applications to generate Social Networks (SN) relating entities occurring in corpora of text. This is due to the fact that there is a growing wealth of knowledge contained in text documents which is difficult to exploit due to their unstructured nature. By generating graphs that subsume the information contained in these documents, we produce a structured view which can be analyzed using Social Network Analysis (SNA) tools. \\

In this section we present the state of the art in the generation of SN from corpora of text and some related applications that use SNA for political analysis. First in section \ref{co-occurrence} we present the entity co-occurrence approach, a simple technique that has been widely used with good results. We then show in \ref{link-characterization} relevant work by the scientific community to characterize the link between two entities in a way that can be used by SNA. We also describe in \ref{topics} methods that can be used to relate entities that despite not being mentioned in the same document might be linked by means of the broader context provided by the whole corpus of documents. Next, we present in section \ref{sna-politics} studies that use SNA for political analysis and that are related to our project. Finally, we succinctly present in \ref{framing} some considerations about how the state of the art was taken into account when making our proposal.\\

\subsection{Entity co-occurrence: a first step towards the creation of SN}\label{co-occurrence}

One of the most widely used approaches to generate SN from text consists in relating entities based on their co-occurrence in a certain context. The underlying assumption of this approach is that if two entities are consistently mentioned together then they are probably related. There are several variants depending on the granularity of the context; one can look for co-occurrence within a certain sentence, paragraph, document or cluster of documents. The choice depends on the amount of data available - finer granularity probably requires more documents to produce more relations - and on the application. \\

The authors of \cite{narcho-networks} propose a method to automatically generate a social network of narco-traffickers in Mexico based on the co-occurrence of names in books about the topic. They do Entity Recognition (ER) to produce a list of entities which is then manually curated and used to determine links between drug dealers. The weight of the relationship is the count of repetitions of the co-occurrences of two entities within a certain distance. They use different network analysis tools to show how the obtained graph closely resembles the different cartels and their chain of command. \\

Similarly, the Joint Research Centre of the European Commission has done extensive work in extracting entities and inter-entities relations from newspapers written in different countries of the European Union. In \cite{jrc-main} we find a summary of their work, which is explained in depth in \cite{jrc-2} and \cite{jrc-3}. Essentially, they look for co-occurrence of entities within previously built clusters of articles that represent a story. They take into account entity coreference and use different heuristics to improve the entity recognition and disambiguation processes. They also produced a formula to measure the strength of a link based not only on the number of co-occurrences but also the frequency of the entities in the clusters and the corpus. By doing this, they aim to weight down relationships in which one of the entities is frequently mentioned, so that only relevant relationships are chosen. \\

Another interesting aspect of their proposal is the use of Wikipedia for validating the obtained graphs. Because we are in presence of a knowledge discovery task for which we do not have a ground truth set, it is difficult to evaluate if the detected links between two entities are meaningful. The definition of a meaningful link is itself not an easy task. The authors of the Joint Research Centre look for the Wikipedia pages of the detected entities and verify if there are links between pages that correspond to the links detected by the system. They define a ``strong'' relationship as one in which there is a reciprocal presence of a link.This allows for the creation of a ground truth set which can be used to evaluate the system with the standard precision and recall metrics. \\

On a different note, the authors of  \cite{google-similarity-measure} present a technique to measure the semantic similarity of two words or phrases by using the Google search engine. They propose a metric based on information distance and Kolmogorov complexity that uses the count of search hits returned by looking up two words individually and together. They show how their approach is useful for distinguishing between colors, numbers, names of paintings and names of books, among others. \\

The main advantage of this approach is that it is able to measure the similarity of two entities based on the whole corpus of documents in the World Wide Web. The drawback is that the number of search hits returned by Google is a gross and often highly inaccurate estimate of the real count. Particularly, it is usually the case that a search with more terms returns a higher count of hits than a search with a subset of these terms. The reason for this is that when adding more terms the search is more fine-grained, allowing for a more refined estimate of documents. More specifically, when having more search terms it is necessary to go deeper through the posting lists which leads to more accurate and larger result estimates. The data centers or the indices used when answering the query also affect the number of expected hits returned. This makes approaches that depend too much on the exact count returned by the search engine unreliable. \\

There are two shortcomings in taking a co-occurrence approach. First, we are often interested in characterizing the link between two entities in terms of strength and meaning to produce richer graphs. It is true that the co-occurrence approach allows a human user to manually inspect the documents in which two entities co-occur. We are however particularly interested in mechanisms that can infer and represent the semantic nature of a link in a way exploitable by social network analysis tools with as little human participation as possible. Second, we are also interested in methods that do not rely on direct co-occurrence within a same document (or a pre-computed cluster of documents), but that can also discover meaningful relationships across a corpus.\\

\subsection{What links two entities? Enriching the SN with the strength and semantics of the relations}\label{link-characterization}

As we previously said, the co-occurrence approach relies on the assumption that if two entities are mentioned in the same context then they are probably related. Co-occurrence may indeed be suggestive that two entities are related, but if there is a relationship we need to characterize that relationship before we can perform SNA. To illustrate this, in the context of our application finding a link between a politician and a third party might indicate that they are closely aligned politically, that they are in opposition, that they participated together in a meeting, that they mentioned each other, among others. Having more information about the found relationships is consequently of great importance to be able to do better analysis.  \\

The efforts of the scientific community to characterize relationships between two entities have been mostly concentrated on Natural Language Processing methods to analyze the context in which the entities co-occur.The authors of \cite{syntactic-template} propose a method which uses dependency trees to learn patterns that relate two entities co-occurring in a sentence according to a pre-defined type of relationship. They work with two examples of relationships: `` support'' and ``meeting''. By working with a small, manually obtained number of seed instances of the relationship - tuples of entities -, they look for sentence co-occurrence in a group of news articles and extract patterns by using their SyntNet GSL algorithm over the dependencies found by a dependency parser. Similarly, in \cite{news-quotation} the authors propose a method to automatically detect quotation relationships in news articles. They aim to do this by also finding linguistic patterns that are usually used when expressing citations: quotation markers and reporting verbs. The authors of \cite{kernel-relation-extraction} illustrate the use of kernel methods for relation extraction. They first produce a shallow parse representation of the texts which is then used by kernels designed specifically to work on parse trees, which have been defined in \cite{tree-kernel}. These kernels are able to implicitly enumerate all possible subtrees of two parse trees, find which are the most common subtrees, weight them and compute a similarity measure based on these. By using a pre-obtained ground truth set, they are able to train classifiers to determine, given two entities and a tree describing the sentence they co-occur in, if the entities have relationships of the type person-affiliation and organization-location.\\

Determining the strength of a relationship is also of interest for SNA applications. To the best of our knowledge, there are no explicit efforts within the scientific community to assess the strength of relations inferred from corpora of text in a systematic way. Most of the applications are mostly concerned with link detection, which is often done by computing and theresholding similarity measures between entities. These similarity measures can be seen as measures of strength; defining a formal method to determine the strength would require however to have previously annotated SNs or a model for specifying what a strong/weak relation is. This is hard, particularly when we take into account the difficulty of manually producing measures of strength that are unbiased and inexpensive. While the question of determining the strength of relationships between two entities has been addressed for online Social Networks in which interactions may be indicative of the strength\cite{xiang2010modeling}, this remains a challenge in the area of SN generation.

\subsection{Going beyond the co-occurrence approach: finding links between entities across documents}\label{topics}

There is also a number of techniques to address the need to find links between entities without depending on their direct co-occurrence within a same document or pre-computed cluster of documents. A widely used approach is automatically inferring topics present in a corpus of text and to verify co-occurrence in articles related to these topics. The authors of  \cite{lda-topics-entities} propose a method that uses Latent Dirichlet Allocation (LDA) to produce a topic model of the documents. In LDA a document is regarded as a finite mixture of topics and represented as a vector in which each component constitutes the probability that the document belongs to a given topic. This model is then used to calculate an entity-entity measure of affinity that is used to find links between entities. The reported results are motivating; the authors show how the use of topics allows to discover more links between entities and to characterize them by means of the topics. They also report however that LDA has however one significant disadvantage: the obtained topics may be hard to interpret and may not be sufficiently semantically cohesive. \\

An alternative to the use of LDA is Latent Semantic Indexing (LSI). By producing a vectorized representation of entities (in which for instance we store information about their co-occurrence in documents), we can use Singular Value Decomposition (SVD) to find a lower dimensional space and estimate the similarity of entities based on latent concepts. This is useful for noise-reduction and for finding semantic relationships between entities. The drawback is that the found relationship may be hard to interpret. The authors of \cite{latent_semantic-index-terrorism} provide an example of the use of LSI for the generation of graphs of terrorists networks.\\

\subsection{SNA and political analysis. Has anyone done this before?}\label{sna-politics}

There are several studies that use SNA for studying the dynamics of power, particularly in legislative bodies. For example, the authors of \cite{fowler2006connecting} studied a graph of co-sponsorship of bills to study interactions between congressmen in the US House of Representatives and found that by using network analysis tools it was possible to find highly influential politicians. Similarly, in \cite{kirkland2011relational} the authors  developed a theory of influence diffusion across a legislative network of relations based on weak and strong links and found patterns useful for determining the success of a bill. In \cite{thomas2006get} we find a proposal to predict the voting of a bill based on speeches made by congressmen. \\

In \cite{chaudhari2014survey} we find a recent survey on the automatic extraction of Policy Networks. They mention several approaches for the generation of SN which we have already mentioned in this chapter (or that are at least closely related to the cited studies) and some other applications using NLP for political science analysis. For instance, in \cite{politician-location} the authors propose a method to automatically extract and characterize relationships between politicians and locations (relations, for instance, of the type (Barack Obama, President, United States); they do so by looking for co-occurrence of persons and locations in web documents, extracting keywords from their context, clustering similar pairs of (Person, Location) and identifying relevant labels. \\

In the survey great attention is paid to the work in  \cite{policy-networks}, as it is the only one, to the best of our knowledge, which specifically addresses the task of automatic Policy Network extraction. The authors of this proposal work with two PNs previously created by experts in a manual, time-consuming process. They evaluate the use of three type of metrics for SN generation that can be produced by using a Web Search Engine:

\begin{itemize}
\item \textbf{\emph{Co-occurrence metrics}}, which measure the degree to which two political actors co-occurr in web pages by looking them up individually and in conjunction in a search engine. Based on the number of results, they produce four metrics of similarity: the \emph{Jaccard Coefficient}, the \emph{Dice Coefficient}, \emph{Mutual Information} and the \emph{Google-based semantic relatedness} \cite{google-similarity-measure}.
\item \textbf{\emph{Text-based metrics}}, which use a vectorial representation of political actors in which components are the frequency of occurrence of words in a certain context of the snippets returned by a search engine. They use cosine similarity to produce a similarity measure from these vectors.
\item \textbf{\emph{Link-based metrics}}, which exploits the hyperlinks of the web pages returned by the search engines to measure the degree of association between actors. The assumption is that if two actors are mentioned in webpages that have links to the same webpages, then they are probably similar. To measure this, they use a version of the \emph{Google-based semantic relatedness}.
\end{itemize}

The manually created SNs contain positive and negative edges, which correspond to relationships of political affinity and aversion, and are annotated with measures of strength. This is particularly useful for understanding how the different methods for graph generations perform. %The authors use Pearson correlation coefficient and the Mean Squared Error to evaluate the difference between their measures of similarity and those of the manually generated SN. 
\\

In general, they obtained better results for when detecting affinity relationships than negative relationships. They also found that using link-based metrics and co-occurrence metrics are the best alternatives for positive relationships while text-based metrics are the best option for negative relationships. When comparing the four proposed measures for co-occurrence based similarity they found that \emph{Mutual Information} is the best alternative for positive links, whereas the \emph{Dice Coefficient} is the best option for negative links. \\
The main difference with our work is that while they work with a predefined list of political actors from the policy networks they use for validation, we are also interested in the discovery of relevant entities and in finding ways to characterize their links. Also, the authors of this proposal did not present any alternatives for automatically inferring the sign of the relationships detected by their system. We use parliamentary data and news articles to address these two needs.\\

\subsection{Framing our work with respect to the state of the art}\label{framing}
 
As we have seen, generating social networks from corpora of text is a topic widely addressed by the scientific community. There are two important considerations with which researchers are concerned: i) discovering the largest number of relationships possible while ensuring the discovered relationships are meaningful and ii) characterizing the relationships between entities in a way exploitable by SNA tools. \\

In the context of our application we are interested in discovering meaningful relationships between politicians and third-parties. We define a meaningful relationship between these two types of entities as a relation of political closeness, meaning that they are affected and have established positions on laws, decrees and other political decisions taken by a parliament. We present in depth our proposal in section \ref{sec:proposal}. However it is worthwhile to present in this section some considerations concerning the state of the art that were taken into account when making our proposal.\\

The task of discovering patterns of political closeness between politicians and third parties is conditioned by the fact that these relationships are usually hidden and unknown by journalists and the public. Entity co-occurrence in a given context thus entails a risk of not revealing all the relevant relationships. Similarly, characterizing the relationships by means of NLP techniques would also require knowledge by the writer of the document of the existence of a relationship between two entities. Finally, discovering topics from the corpus of documents could also lead to using topics that do not correspond to political themes. \\

To address this, we use bills as the cornerstone that allows us to link politicians and third parties. By using bills, we can model interpretable and semantically cohesive topics that allows us to address the two considerations mentioned earlier in this section. We can specifically i) detect meaningful links between entities across documents - thus increasing the precision and recall of our system in terms of an imaginary ground truth set - and ii) characterize the found links by using the bill that led to the discovery of the link.\\ 

CONTINUAR ESTO CUANDO ESTE MEJOR EXPLICADA LA PROPUESTA.

\newpage
\section{Implementation}\label{sec:implementation}

In this chapter give some relevant implementation details, along with a diagram showing the pipeline of the system. \\

\subsection{Modelling topics after bills }\label{subsec:topic}

\subsection{Entity Recognition and Preprocessing}\label{subsec:getting-entities}

After finding articles related to bills, they are analyzed to find entities that are in turn related to the bills. For this purpose we use MITIE, a state of art tool Named Entity Recognition tool created in the MIT. Given a document, MITIE identifies substrings that contain possible named entities and tags them as \emph{Organization, Location, Person} or \emph{Miscellaneous}.\\

Before carrying on the detected entities need to be pre-processed before they can be used. The names found may i) not be correctly delimited (resulting in truncated names or in names that contain excess text) ii) be ambiguous (an entity may have more than one name and a name may refer to more than one entity) and iii) may be noise and not refer to a real entity. \\

Name disambiguation is in itself a complex and interesting research topic which escapes the aim of this study. We have however implemented some heuristics we briefly describe below:\\

\begin{enumerate}

\item \textbf{Entity Normalization}: entities are brought to a canonical form to address spelling variations. This involves i) punctuation sign removal; ii) double, leading and trailing whitespaces removal; iii) leading and trailing stop words removal; iv) string camelization (all characters are put in lowercase except for the first letter of every word, which is in uppercase). 

\item \textbf{Mapping organization initials to the whole name}: when an organization name is detected we aim to detect if there are any other names composed by its initials. Specifically, we aim to exploit a widely found pattern: organizations often have the full name followed by the initials inside parenthesis. 

\item \textbf{Mapping partial names with full names}: when person names are detected we classify them into \emph{full names} (containing more than 1 word) and \emph{short names} (containing one word, which is possible the last or first name of the full name). We then link short names with the nearest, previous full name such that the short name is contained inside the full name.

\item \textbf{Expanding names based on the news corpus}: to address the issue of truncated names (eg 'Word Life Fund' may be truncated and processed as 'World and 'Life Fund')  we: i) look up every name and it's surrounding context in the corpus ii) extract sentences in the top articles and iii) find the longest substring matching these sentences. 
\end{enumerate}

By doing this we find a list of entities for which we have a list of aliases and a list of tags counts. After doing this, we choose for every entity the tag with the highest frequency as the type of the entity. \\



\newpage
\section{Results}\label{sec:results}

Results go here.

\newpage
\section{Conclusions}\label{sec:conclusions}

Conclusions come here.

Future work:
- Validation by experts
- Time component

\newpage

\references

 
\appendixECD


\end{document}

