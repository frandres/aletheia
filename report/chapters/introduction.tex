\section{Introduction}\label{sec:intro}

Modern law-making is characterized by the indirect participation of third-parties in the legislative process to attempt to influence the contents and outcome of the discussions according to their interests. This process, known as \emph{lobbying}, is regularly done by companies, Non-Governmental Organizations (NGOs) and in some cases citizens and foreign governments. For instance, when a law concerning human rights is being discussed in a parliament we can expect human rights NGOs to contact politicians to attempt to push their agenda. \\

Most of the times citizens are completely unaware of the lobbying activities of their representatives. This is due to the fact that it is hard to closely monitor the activities of politicians, to understand the intricacies behind the drafting of bills and to grasp the implications they have on organizations and society. Knowing how decisions makers are influenced is however of great interest for politologists, journalists and voters in general. By tracking these relations we would be able to understand better the way decisions are made by government, have a better sense of who politicians are, who they serve and what their agenda is, and in some cases fight corruption. \\

Recently, there has been considerable work in the development of data mining tools to analyze the way legislative bodies work. Social Network Analysis (SNA) has been widely used to understand the dynamics of complex social systems and is a natural tool for modelling the dynamics of power in a parliament. There is a wealth of knowledge to be extracted from parliamentary data to model relationships between politicians, identify key players and sub-communities, predict the voting of bills and in general gain a better understanding of the legislative process. However, finding political relations between representatives and third party actors, who often do not appear in bills or in the transcripts of debates, remains a challenge. How can we automatically detect that a politician and a company might be aligned from information available to the public? \\

News articles contain a sizable amount of information about what happens in a given country and about its relevant actors. Because of this, there is plenty of work by the scientific community to develop methods to automatically extract useful knowledge from news articles. The development of applications to automatically generate Social Networks (SN) from corpora of news articles is particularly interesting in our context. This has been done extensively in the past in different areas including defense, counter-terrorism, news summarization and crime prevention. The working assumption is that by detecting patterns of co-occurrence of two entities in text we can establish that they are in some way related.  \\

In this study we use that assumption and formulate the hypothesis that by analyzing and combining parliamentary data and news articles we can generate graphs descriptive of the political closeness of politicians and relevant actors of a country. Our purpose is consequently to propose a method to automatically detect patterns that could be indicative of a lobbying activity. To do this, we automatically analyze news articles, text of bills discussed in a parliament and the transcripts of the debates. We show that it is possible to i) track the participation and the position of a politician with respect to a law and ii) find entities in news articles which could be directly related to the contents of a law. Laws thus can then be used as a cornerstone for detecting relationships between politicians and third-party actors. These relationships can then be used to build a SN which can be analysed using SNA tools. In this document we present our proposal and show the results obtained by its application in the context of the Parliament of Catalonia. \\

It is important to clarify that despite being motivated by the detection of patterns that can be suggestive of a lobbying process, in practice we aim to find relationships of political similarity or dissimilarity between two entities. The relations found by our method do not necessarily imply that the two actors are in direct liaison;  to do so we would need to closely monitor all the activities of politicians and organizations to verify with whom they are in contact. This is naturally unreasonable due to privacy considerations. \\

We aim to detect links between entities that indicate that they are both related to a particular political decision - in our case bills -, from which we could then establish political affinity or aversion. Naturally, if two entities have highly similar political views in a broad set of issues then one can believe that they could be collaborating. This could be verified by investigative journalists and by considering other sources of information like donations information, speeches, etcetera. Regardless of the verification of a lobbying process, having a graph relating politicians and third parties is in itself useful to understand the political landscape of a country. In this study we also show how our proposal is useful for political analysis, besides from the interest in the detection of lobbying activities. \\

The rest of this document is organized as follows: in chapter \ref{sec:related-works} we present the state of the art in the automatic generation of SN from corpora of text and in the use of SNA in the context of political science. In chapter\ref{sec:proposal} we present our proposal and in chapter \ref{sec:implementation} we present some implementation considerations deemed relevant. Next, we present the obtained results in section \ref{sec:results} along with some interesting applications of our method. Finally, we present in chapter \ref{sec:conclusions} the conclusions of our work along with some suggestions for further work.

