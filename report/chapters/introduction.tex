\section{Introduction}\label{sec:intro}

Political decision-making is characterized by the indirect participation of third-parties to attempt to influence the contents and outcome of the discussions according to their interests. This process, known as \emph{lobbying}, is regularly done by companies, Non-Governmental Organizations (NGOs) and in some cases citizens and foreign governments. For instance, when a law concerning human rights is being discussed in a parliament we can expect human rights NGOs to contact politicians to attempt to push their agenda. \\

Most of the times citizens are completely unaware of the lobbying activities of their representatives. This is due to the fact that it is hard to closely monitor the activities of politicians, which in many cases are opaque to people, to understand the intricacies behind their decisions and to grasp the implications they have on organizations and society. Knowing how decision makers are influenced is however of great interest for politologists, journalists and voters in general. By tracking these relations we would be able to better understand the way decisions are made by a government, have a better sense of who politicians are, who they serve and what their agenda is, and in some cases fight corruption. \\

Among the different powers of the classical liberal democracies, the Legislative Branch is particularly interesting in the context of lobbying and policy making analysis. This is due to the fact that parliaments are responsible for the writing and passing of laws and regulations, the ratification of international treaties and the oversight of other branches of government. This makes them particularly influential in the shaping of a society and its institutions. Because of this, there has been considerable work in the development of data mining tools for understanding the way legislative bodies work.\\ 

Social Network Analysis (SNA) has been widely used to understand the dynamics of complex social systems. This makes SNA a natural tool for modelling the dynamics of power in a parliament. There is a wealth of knowledge to be extracted from parliamentary data to model relationships between politicians, identify key players and sub-communities, predict the voting of bills and in general gain a better understanding of the legislative process. However, finding political relations between representatives and third party actors, who often do not appear in bills or in the transcripts of debates, remains a challenge. \\

News articles on the other hand contain a sizable amount of information about what happens in a given country and about its relevant actors. Because of this, there is plenty of work by the scientific community to develop methods to automatically extract useful knowledge from news articles. The development of applications to automatically generate Social Networks (SN) from corpora of news articles is particularly interesting in our context. This has been done extensively in the past in different areas including defense, counter-terrorism, news summarization and crime prevention. The working assumption is that by detecting patterns of co-occurrence of two entities in text we can establish that they are in some way related.  \\

In this study we used that assumption and formulated the hypothesis that by analyzing and combining parliamentary data and news articles we can generate SN for the members of a Legislative Body and entities that are related to these politicians. Our purpose is consequently to propose a method to automatically detect links descriptive of the political closeness of politicians and relevant actors of a country, or in other words, patterns indicative of a possible lobbying activity. To do this, we automatically analyze news articles, the text of bills discussed in a parliament and the transcripts of the debates.   In this document we present our proposal and show the results obtained by its application in the context of the Parliament of Catalonia. \\

The rest of this document is organized as follows: in chapter \ref{sec:problem_definition} we present a more detailed definition of our problem. This is followed by chapter \ref{sec:related-works}, in which we present the state of the art in the automatic generation of SN from corpora of text and in the use of SNA in the context of political science. In chapter \ref{sec:implementation} we describe our proposal and present some relevant considerations about its implementation. Next, we present the obtained results in section \ref{sec:results} by means of case studies. Finally, we present in chapter \ref{sec:conclusions} the conclusions of our work along with some suggestions for further work.

