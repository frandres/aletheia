\section*{Hosting institution}

This Master's thesis was done with the support and supervision of the DAMA and LARCA Research Groups at Universitat Polit\`ecnica de Catalunya (UPC). 

\subsection*{DAta MAnagement group - DAMA}

DAMA-UPC is part of the Computer Architecture Department (DAC). The main research topics of DAMA-UPC are oriented to performance, exploration and quality in data management, focusing particularly on large data volumes. Specifically, they have investigated the creation of new data structures, algorithms, methods and applications in the area of Data Management that make it easier to manipulate large amounts of data.\\

DAMA-UPC is a member of Tecnio since 2005. Tecnio is an initiative of ACC10, the Agency for Innovation and Internationalization of the Catalan Enterprise, belonging to Generalitat de Catalunya. It is supported by Generalitat de Catalunya as a Consolidated Research Group (SGR-1187) and by the Ministry of Education and Science of Spain. Since its creation it has worked with important technological partners such as IBM, Oracle technologies, the Ministry of Cience and Innovation, the Health Department of the Generalitat de Catalunya, among others; and is current participating in four European Research Projects. \\

Moreover, DAMA-UPC has been a pioneer in scientific collaboration, creating and organizing for three consecutive years the Workshop on Graph-based Technologies and Applications (Graph-TA), organizing the 17th International Database Engineering \& Applications Symposium (IDEAS 2013) and the First University-Industry Meeting on Graph Databases (UIM-GDB). The work of DAMA has also give birth to Sparksee, a graph database, and Sparsity Technologies.\\

\subsection*{Laboratory of Relational Algorithmics, Complexity and Learnability - LARCA}

LARCA is an international research group working on data mining, machine learning, data analysis, and mathematical linguistics. They typically approach problems from sound mathematical principles, using modelling tools and techniques from algorithmics, computational complexity, automata theory, logic, discrete mathematics, statistics, and dynamic systems. \\

LARCA has also participated in several European research projects and collaborated with several industrial partners including Gas Natural Fenosa, Xopie, 4dLife, Urbiotica and Vingenia. They also foster collaboration with other research groups in and outside of Spain, including ALBCOM: Algorithms, Computational Biology, Complexity and Formal Methods Research Group at CS-UPC; NLP: Natural Language Processing Research Group at CS-UPC; IAIA: Investigacion y Aplicaciones en Inteligencia Artificial group at U. M\'alaga; CCG: Computational Complexity Group at U. Zaragoza; MIDAS: Spanish Network on Data Mining and Learning; Machine Learning Group at the University of Waikato, New Zealand and the Real and functional Analysis Research Group of Universitat de Barcelona. \footnote{Text partially taken from \url{www.dama.upc.edu} and \url{https://recerca.upc.edu/larca/}}
